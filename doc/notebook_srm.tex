% Autor Tobias Erbsland.
% Veröffentlicht unter The GNU Free Documentation License.
% modifiziert von Hannes Röst
% Version 1.2
%
% A. DOKUMENTKLASSE
% ---------------------------------------------------------------------------
%

%
%  1. Definieren der Dokumentklasse.
%     Wir verwenden die KOMA-Script Klasse 'scr...' 
%
\documentclass[
  pdftex,           %PDFTex verwenden da wir ausschliesslich ein PDF erzeugen.
  a4paper,          %Wir verwenden A4 Papier.
  oneside,          %Einseitiger Druck.
  12pt,             %12 Punkte, besser geeignet für A4.
  halfparskip,      %Halbe Zeile Abstand zwischen Absätzen.
  %chapterprefix,    %Kapitel mit 'Kapitel' anschreiben.
  %headsepline,      %Linie nach Kopfzeile.
  %footsepline,      %Linie vor Fusszeile.
  bibtotocnumbered, %Literaturverzeichnis im Inhaltsverz. nummeriert einfügen.
  idxtotoc          %Index ins Inhaltsverzeichnis einfügen.
]{scrartcl}


%
%  2. Festlegen der Zeichencodierung des Dokuments und des Zeichensatzes.
%     Wir verwenden 'UTF-8' für das Dokument,
%     und die 'T1' codierung für die Schrift.
%
\usepackage[T1]{fontenc}
\usepackage[utf8]{inputenc}

%  3. Packet für die Index-Erstellung laden.
%     makeindex ist ein Programm zur Erstellung von Stichwortverzeichnissen für LaTeX-Dokumente
%
\usepackage{makeidx}

%
%  4. Neue deutsche Rechtschreibung mit 'ngerman'. Es kommt es zu einer Übersetzung im Dokument
%     (Inhaltsverzeichnis statt table of contents, Daten). Außerdem wird nach deutscher
%     Rechtschreibung getrennt.
\usepackage[ngerman, english]{babel}


%
%  5. Paket für Anführungszeichen laden.
%     Wir setzen den Stil auf 'swiss', und verwenden so die Schweizer Anführungszeichen.
%
%\usepackage[style=swiss]{csquotes}


%
%  6. Paket für erweiterte Tabelleneigenschaften.
%
\usepackage{array}

%
%  7. Paket um Grafiken im Dokument einbetten zu können.
%     Im PDF sind GIF, PNG, und PDF Grafiken möglich.
%
\usepackage{graphicx}

%
%  8. Pakete für mathematischen Textsatz.
%     siehe dazu auch http://www.andy-roberts.net/misc/latex/latextutorial10.html
\usepackage{amsmath}
\usepackage{amssymb}
\usepackage{dsfont}
\usepackage{mathtools}

%
%  9. Paket um Textteile drehen zu können.
%     \begin{turn}{Winkel} Text \end{turn}
\usepackage{rotating}

%
% 10. Paket für Farben an verschieden Stellen. 
%     Wir definieren zusätzliche benannte Farben.
%     Mit dem Befehl \pagecolor{Farbe} wird die Hintergrundfarbe, und mit dem 
%     Befehl \color{Farbe} die Textfarbe bestimmt. 
%     http://www.willemer.de/informatik/text/texcolor.htm
\usepackage{color}

%
% 11. Paket für spezielle PDF features.
%
%     pdftitle     Titel des PDF Dokuments.
%     pdfauthor    Autor des PDF Dokuments.
%     pdfsubject   Thema des PDF Dokuments.
%     pdfcreator   Erzeuger des PDF Dokuments.
%     pdfkeywords  Schlüsselwörter für das PDF.
%     pdfpagemode  Inhaltsverzeichnis anzeigen beim Öffnen
%     pdfdisplaydoctitle     Dokumenttitel statt Dateiname anzeigen.
%     pdflang      Sprache des Dokuments.
\usepackage[
  pdfpagemode=UseOutlines,
  pdfdisplaydoctitle=true,
  %pdflang=en,
  pdfauthor={Hannes Roest},
  %pdftitle={Hannes Roest},
  colorlinks=true,
  linkcolor=black,
  citecolor=black,
  filecolor=black,
  pagecolor=black,
  urlcolor=black
%  frenchlinks=true
]{hyperref}

%
% 12. Font wenn die Schrift nicht schön aussieht, verwende eine andere
%     siehe auch http://www.matthiaspospiech.de/latex/vorlagen/allgemein/preambel/fonts/
%\usepackage{mathptmx}
%\usepackage[scaled=.90]{helvet}
%\usepackage{courier}


%
% 13. Line Spacing
%     This might give better (and might give worse) results

%\usepackage{setspace}
%\singlespacing        %% 1-spacing (default)
%\onehalfspacing       %% 1,5-spacing
%\doublespacing        %% 2-spacing


%
% 14. Kopf und Fusszeilen
%

\usepackage{fancyhdr}
\pagestyle{fancy}

\fancyhf{} % aktuelle header and footer löschen
% Buchstabencodes für \fancyhead und \fancyfoot sind:
% E     Even page
% O     Odd page
% L     Left
% C     Center
% R     Right
% Können gemixt werden, also auch: \fancyhead[RO,RE]{irgendetwas} für 
%          "rechts oben auf geraden und ungeraden seiten"
% "Gerade" und "ungerade" Seite werden unterschieden für Bücher, wo man die
% Seitenzahl immer aussen haben möchte.
%\renewcommand{\chaptermark}[1]{\markboth{#1}{}}
\renewcommand{\sectionmark}[1]{\markright{#1}{}}
\fancyhead[L]{\thechapter \phantom{L}  \leftmark}
\fancyhead[C]{}
\fancyhead[R]{\thesection \phantom{L} \rightmark}
\fancyfoot[L]{}
\fancyfoot[C]{\thepage }
%Linie vor/nach Fuss/Kopfzeile
\renewcommand{\headrulewidth}{0.0pt} 
\renewcommand{\footrulewidth}{0.0pt}
%\addtolength{\headheight}{0.5pt} % platz machen für die linie
%\addtolength{\footheight}{0.5pt} % platz machen für die linie
\fancypagestyle{plain}{%
       \fancyhead{} % kopfzeilen auf leeren seiten loswerden
       \renewcommand{\headrulewidth}{0pt} % ... und auch die linie
       \renewcommand{\footrulewidth}{0pt} % ... und auch die linie
}

%%
%% 15. Schöne Boxen
%%
%\usepackage{fancybox}

%% File Extensions of Graphics %%%%%%%%%%%%%%%%%%%%%%%%%%%%%%
%% ==> This enables you to omit the file extension of a graphic.
%% ==> "\includegraphics{title.eps}" becomes "\includegraphics{title}".
%% ==> If you create 2 graphics with same content (but different file types)
%% ==> "title.eps" and "title.pdf", only the file processable by
%% ==> your compiler will be used.
%% ==> pdfLaTeX uses "title.pdf". LaTeX uses "title.eps".
\DeclareGraphicsExtensions{.pdf,.jpg,.png}

%% Martina's preferences %%%%%%%%%%%%%%%%%%%%%%%%%%%%%%%%%%%%
%\setlength{\parskip}{6pt} %Legt den Abstand zwischen den nachfolgenden Absätzen fest. 
\usepackage{url} %besser mit hypperref
%\usepackage[paren, plain]{fancyref}
%\newcommand{\species}[1]{\textsl{#1}}  %%formating species names
%\setcounter{fignumdepth}{1} %%How can figurs be numbered independently of the chapter


%% Other Packages %%%%%%%%%%%%%%%%%%%%%%%%%%%%%%%%%%%%%%%%%%%
%\usepackage{a4wide} %%Smaller margins = more text per page.
%\usepackage{longtable} %%For tables, that exceed one page


%%Attention: the correct dash is the following: -


\usepackage{gensymb}
\usepackage{hyphenat}
\usepackage{subfig}
\usepackage{verbatim} 

\usepackage{booktabs} %make lines in tables (midrule, toprule etc)
\newcommand*\maketablespace{ \addlinespace[15pt] }

%\setcounter{tocdepth}{3}
\begin{document}

\author{Hannes Röst}
\title{Lab Notebook SRM clashes}
\maketitle

\tableofcontents

\section{Peptides between 400 and 1200 Da}

\subsection{Introduction}


See table~\ref{tab:yeast_pepprot} for the peptide and protein analysis. It
shows how many peptides and proteins are visible in a window of 400 to 1200 Da
and how many of those have 2+ and 3+ charged parent ions in that window. I
calculated it for all peptides and for the unique peptides only.

\begin{table}[h]

\centering
\caption[Yeast peptide and protein distribution.]
{\textbf{Yeast peptide and protein distribution.}
The number of proteins and peptides in yeast. Shown are all, those which have
parent ions between 400 and 1200 Da and finally those which additionally also
have both 2+ and 3+ parent ions in that range. \newline
The top table show all peptides, the bottom table shows only those which occur
once in the genome.
}
\label{tab:yeast_pepprot}

\begin{tabular}{ l c c }
\maketablespace
Description & Peptides  & Proteins \\
\toprule
All & 137\,161 & 6602 \\
400-1200 & 130\,734 & 6593 \\
400-1200, 2+ and 3+ parent ions & \phantom{1}62\,663 & 6422\\
\toprule
All & 134\,412 & 6437 \\
400-1200 & 128\,094 & 6419 \\
400-1200, 2+ and 3+ parent ions & \phantom{1}61\,530 & 6180\\

\end{tabular}

\begin{comment}
%%%%%%%%%%%%%%%%%%%%%%%%%%%%%%%%%%%%%%%%%%%%%%%%%%%%%%%%%%%%%%%%%%%%%%%%%%%
%%#SQL to create this data
drop table tmptbl;
create temporary table tmptbl as
select count(*) as occ , srmPeptides_yeast.peptide_key from hroest.srmPeptides_yeast 
inner join ddb.peptide on srmPeptides_yeast.peptide_key = peptide.id 
inner join ddb.peptideOrganism on peptideOrganism.peptide_key = peptide.id
where q1 > 400 and q1 < 1200 
and genome_occurence = 1
group by srmPeptides_yeast.peptide_key
;
alter table tmptbl add index(peptide_key);

select count(*) from tmptbl
#inner join ddb.protPepLink on protPepLink.peptide_key = tmptbl.peptide_key
#group by protein_key
;

select count(*) from tmptbl
#inner join ddb.protPepLink on protPepLink.peptide_key = tmptbl.peptide_key
where occ = 2
#group by protein_key
;



#
##
#
# -- check whether we really filtered out non-unique peptides
create temporary table unique_yeast_proteins as
select protein_key from ddb.peptide 
#select count(*) from ddb.peptide 
inner join ddb.protPepLink on protPepLink.peptide_key = peptide.id
inner join ddb.peptideOrganism on peptideOrganism.peptide_key = peptide.id
where experiment_key = 3120 
and genome_occurence = 1
group by protein_key
;

drop table non_unique_yeast_proteins;
create temporary table non_unique_yeast_proteins as
select distinct protein_key from ddb.peptide 
inner join ddb.protPepLink on protPepLink.peptide_key = peptide.id
where experiment_key = 3120 
and protein_key not in (select * from unique_yeast_proteins)
;

#see how many peptides each of the non unique proteins has
select protPepLink.protein_key, count(*) from
non_unique_yeast_proteins 
inner join ddb.protPepLink on protPepLink.protein_key = non_unique_yeast_proteins.protein_key
group by protPepLink.protein_key;

\end{comment}

\end{table}

I also calculated the number of 4+ and 5+ charged parent ions that would fall
into the range of 400 to 1200, see Table~\ref{tab:yeast_pepprot_ions}.

\begin{table}[h]

\centering
\caption[Yeast peptides in 400 to 1200 Da.]
{\textbf{Yeast peptides in 400 to 1200 Da.} Grouped by parent ion charge.  }
\label{tab:yeast_pepprot_ions}

\begin{tabular}{ r r }
\maketablespace
Parent ion charge & Peptides   \\
\toprule
1 & 100\,535  \\
2 & 111\,648  \\
3 &  82\,027  \\
4 &  57\,837  \\
5 &  40\,350  \\
6 &  28\,166  \\
7 &  19\,500  \\

\end{tabular}
%#number of 3+/4+ parent ions in 400-1200 range
%SET @CHARGE = 7;
%select count(*) from ddb.peptide
%where experiment_key = 3131
%and (molecular_weight + @CHARGE) / @CHARGE > 400 
%and (molecular_weight + @CHARGE) / @CHARGE < 1200
%;

\end{table}

In total ther are 128\,094 parent ions in the range of 400 to 1200 Da and they
have  1\,690\,316 transitions.

All the following data was created with these parameters: 
Searched 2+ parent and 1+ fragment ion against a background of 2+ parent (1+,2+
fragment) and 3+ parent (1+,2+ fragment). The background was the ensembl yeast
genome release, release 57.

The SSRCalc window was 2 arbitrary units, Q1 variable and Q3 was 10 ppm.

Only ions from peptides with a molecular mass between 800\,Da and 5000\,Da 
and transitions between 400 and 1200\,Th were included. 
%\subsection{50 Da 10ppm}
%
%%Non unique transitions are 0.51060038478~\% or 863\,076
%Nonunique / Total transitions : 863076.0 / 1690316.0 = 0.51060038478
%
%Percentage of collisions below 1 ppm: 74.79~\%
%
%See Figure~\ref{fig:400range.50Da_10ppm}.
%
%\begin{figure}
%
%\center
%\subfloat[1][Unique transition distribution]{\includegraphics[width=0.6\textwidth, angle=-90]{/home/hroest/srm_clashes/results/pedro/yeast_True_False_20_500_100ppm_range400to1200.pdf}} \\
%\subfloat[1][Cumulative unique transition distribution]{\includegraphics[width=0.6\textwidth, angle=-90]{/home/hroest/srm_clashes/results/pedro/yeast_True_False_20_500_100ppm_range400to1200_cum.pdf}} 
%
%\caption{ \textbf{Percentage unique transitions in the tryptic yeast peptidome.}
%The number of peptides is plotted against the percentage of unique transitions
%per peptide (top). Below, a cumulative distribution is shown.
%}
%\label{fig:400range.50Da_10ppm}
%\end{figure}
%
%\subsection{25 Da 10ppm}
%
%Nonunique / Total transitions : 602972.0 / 1690316.0 = 0.356721465099
%
%Percentage of collisions below 1 ppm: 73.63~\%
%
%See Figure~\ref{fig:400range.25Da_10ppm}.
%
%\begin{figure}
%
%\center
%\subfloat[1][Unique transition distribution]{\includegraphics[width=0.6\textwidth, angle=-90]{/home/hroest/srm_clashes/results/pedro/yeast_True_False_20_250_100ppm_range400to1200.pdf}} \\
%\subfloat[1][Cumulative unique transition distribution]{\includegraphics[width=0.6\textwidth, angle=-90]{/home/hroest/srm_clashes/results/pedro/yeast_True_False_20_250_100ppm_range400to1200_cum.pdf}} 
%
%\caption{ \textbf{Percentage unique transitions in the tryptic yeast peptidome.}
%The number of peptides is plotted against the percentage of unique transitions
%per peptide (top). Below, a cumulative distribution is shown.
%}
%\label{fig:400range.25Da_10ppm}
%\end{figure}
%
%\subsection{20 Da 10ppm}
%
%Nonunique / Total transitions : 536847 / 1690316 = 0.317601560892
%
%Percentage of collisions below 1 ppm: 19.80~\%
%
%%See Figure~\ref{fig:400range.15Da_10ppm}.
%
%\subsection{15 Da 10ppm}
%
%Nonunique / Total transitions : 448789 / 1690316 = 0.265505976397
%
%Percentage of collisions below 1 ppm: 16.59~\%
%
%
%See Figure~\ref{fig:400range.15Da_10ppm}.
%
%\begin{figure}
%
%\center
%\subfloat[1][Unique transition distribution]{\includegraphics[width=0.6\textwidth, angle=-90]{/home/hroest/srm_clashes/results/pedro/yeast_True_False_20_150_100ppm_range400to1200.pdf}} \\
%\subfloat[1][Cumulative unique transition distribution]{\includegraphics[width=0.6\textwidth, angle=-90]{/home/hroest/srm_clashes/results/pedro/yeast_True_False_20_150_100ppm_range400to1200_cum.pdf}} 
%
%\caption{ \textbf{Percentage unique transitions in the tryptic yeast peptidome.}
%The number of peptides is plotted against the percentage of unique transitions
%per peptide (top). Below, a cumulative distribution is shown.
%}
%\label{fig:400range.15Da_10ppm}
%\end{figure}
%
%
%\subsection{9 Da 10pp}
%
%Nonunique / Total transitions : 310193.0 / 1690316.0 = 0.18351184039
%
%Percentage of collisions below 1 ppm: 74.29~\%
%
%
%See Figure~\ref{fig:400range.9Da_10ppm}.
%
%\begin{figure}
%
%\center
%\subfloat[1][Unique transition distribution]{\includegraphics[width=0.6\textwidth, angle=-90]{/home/hroest/srm_clashes/results/pedro/yeast_True_False_20_90_100ppm_range400to1200.pdf}} \\
%\subfloat[1][Cumulative unique transition distribution]{\includegraphics[width=0.6\textwidth, angle=-90]{/home/hroest/srm_clashes/results/pedro/yeast_True_False_20_90_100ppm_range400to1200_cum.pdf}} 
%
%\caption{ \textbf{Percentage unique transitions in the tryptic yeast peptidome.}
%The number of peptides is plotted against the percentage of unique transitions
%per peptide (top). Below, a cumulative distribution is shown.
%}
%\label{fig:400range.9Da_10ppm}
%\end{figure}
%
%\subsection{1 Da 10ppm}
%
%Nonunique / Total transitions : 86612.0 / 1690316.0 = 0.0512401231486
%
%Percentage of collisions below 1 ppm: 79.77~\%
%
%
%See Figure~\ref{fig:400range.1Da_10ppm}.
%
%\begin{figure}
%
%\center
%\subfloat[1][Unique transition distribution]{\includegraphics[width=0.6\textwidth, angle=-90]{/home/hroest/srm_clashes/results/pedro/yeast_True_False_20_10_100ppm_range400to1200.pdf}} \\
%\subfloat[1][Cumulative unique transition distribution]{\includegraphics[width=0.6\textwidth, angle=-90]{/home/hroest/srm_clashes/results/pedro/yeast_True_False_20_10_100ppm_range400to1200_cum.pdf}} 
%
%\caption{ \textbf{Percentage unique transitions in the tryptic yeast peptidome.}
%The number of peptides is plotted against the percentage of unique transitions
%per peptide (top). Below, a cumulative distribution is shown.
%}
%\label{fig:400range.1Da_10ppm}
%\end{figure}
%
%\subsection{15 Da 20ppm}
%\subsection{20 Da 20ppm}
%
%Nonunique / Total transitions : 645366 / 1690316 = 0.381801982588
%
%Percentage of collisions below 1 ppm: 20.39~\%
%
%%See Figure~\ref{fig:400range.15Da_10ppm}.
%
%\subsection{Traditional SRM: 1.2 Da, 1.0 Da}
%Nonunique / Total transitions : 685310 / 1690316 = 0.405433066953
%Percentage of collisions below 1 ppm: 22.16~\%
%
%\subsection{Traditional SRM: 1 Da, 1 Da}
%
%This is the comparison to traditional SRM with Q1 of 1 Da and Q3 of 1 Da.
%
%Nonunique / Total transitions : 579367 / 1690316 = 0.342756620656
%
%Percentage of collisions below 1 ppm: 20.48~\%
%
%See Figure~\ref{fig:400range.10Da_1Da}.
%
%\begin{figure}
%
%\center
%\subfloat[1][Unique transition distribution]{\includegraphics[width=0.6\textwidth, angle=-90]{/home/hroest/srm_clashes/results/pedro/yeast_True_False_20_10_10_range400to1200.pdf}} \\
%\subfloat[1][Cumulative unique transition distribution]{\includegraphics[width=0.6\textwidth, angle=-90]{/home/hroest/srm_clashes/results/pedro/yeast_True_False_20_10_10_range400to1200_cum.pdf}} 
%
%\caption{ \textbf{Percentage unique transitions in the tryptic yeast peptidome.}
%The number of peptides is plotted against the percentage of unique transitions
%per peptide (top). Below, a cumulative distribution is shown.
%}
%\label{fig:400range.10Da_1Da}
%\end{figure}
%
%\subsection{Traditional SRM: 0.7 Da, 1 Da}
%
%This is the comparison to traditional SRM with Q1 of 0.7 Da and Q3 of 1 Da.
%
%Nonunique / Total transitions : 418245.0 / 1690316.0 = 0.247435982384
%
%Percentage of collisions below 1 ppm: 20.04~\%
%
%See Figure~\ref{fig:400range.07Da_1Da}.
%
%\begin{figure}
%
%\center
%\subfloat[1][Unique transition distribution]{\includegraphics[width=0.6\textwidth, angle=-90]{/home/hroest/srm_clashes/results/pedro/yeast_True_False_20_7_10_range400to1200.pdf}} \\
%\subfloat[1][Cumulative unique transition distribution]{\includegraphics[width=0.6\textwidth, angle=-90]{/home/hroest/srm_clashes/results/pedro/yeast_True_False_20_7_10_range400to1200_cum.pdf}} 
%
%\caption{ \textbf{Percentage unique transitions in the tryptic yeast peptidome.}
%The number of peptides is plotted against the percentage of unique transitions
%per peptide (top). Below, a cumulative distribution is shown.
%}
%\label{fig:400range.07Da_1Da}
%\end{figure}
%
%\subsection{Traditional SRM: 0.5 Da, 1.0 Da}
%
%Nonunique / Total transitions : 378720 / 1690316 = 0.224052780663
%Percentage of collisions below 1 ppm: 14.80~\%
%
%\subsection{Traditional SRM: 0.2 Da, 1.0 Da}
%
%Nonunique / Total transitions : 313089 / 1690316 = 0.185225129502
%
%Percentage of collisions below 1 ppm: 11.88~\%
%
%\subsection{Traditional SRM: 0.1 Da, 0.1 Da}
%Nonunique / Total transitions : 177012 / 1690316 = 0.104721247388
%Percentage of collisions below 1 ppm: 6.44~\%

\subsection{Q1 and Q3 distributions}

See Figure~\ref{fig:400range.dist_closest} for the Q1 and Q3 distribution of the closest hit and 
Figure~\ref{fig:400range.dist_all} for the Q1 and Q3 distribution of all hits.

\begin{figure}

\center
\subfloat[2][Q3 distance distribution]{\includegraphics[width=0.6\textwidth, angle=-90]{/home/hroest/srm_clashes/results/pedro/yeast_True_False_20_500_1000ppm_range400to1200_q3distr_ppm.pdf}} \\
\subfloat[2][Q1 distance distribution]{\includegraphics[width=0.6\textwidth, angle=-90]{/home/hroest/srm_clashes/results/pedro/yeast_True_False_20_500_1000ppm_range400to1200_q1distr.pdf}}

\caption{
Q1 and Q3 difference distributions of the closest hit for a search using 50 Da and 100 ppm windows.
}
\label{fig:400range.dist_closest}
\end{figure}

\begin{figure}

\center
\subfloat[2][Q3 distance distribution]{\includegraphics[width=0.6\textwidth, angle=-90]{/home/hroest/srm_clashes/results/pedro/yeast_True_False_20_500_1000ppm_range400to1200_q3all_distr_ppm.pdf}} \\
\subfloat[2][Q1 distance distribution]{\includegraphics[width=0.6\textwidth, angle=-90]{/home/hroest/srm_clashes/results/pedro/yeast_True_False_20_500_1000ppm_range400to1200_q1all_distr.pdf}}

\caption{
Q1 and Q3 difference distributions of all hits for a search using 50 Da and 100 ppm windows.
}
\label{fig:400range.dist_all}
\end{figure}

\begin{figure}


\center
\subfloat[2][Q3 distance distribution (zoom)]{\includegraphics[width=0.6\textwidth, angle=-90]{/home/hroest/srm_clashes/results/pedro/yeast_True_False_20_20_20_range400to1200_q3all_distr_ppm_zomm500.pdf}} \\
\subfloat[2][Q3 distance distribution]{\includegraphics[width=0.6\textwidth, angle=-90]{/home/hroest/srm_clashes/results/pedro/yeast_True_False_20_20_20_range400to1200_q3all_distr_ppm_zomm3000.pdf}}

\caption{
Q3 difference distributions of all hits for a search using 2 Da and 2 Da windows.
}
\label{fig:400range.dist_all_largeQ3}
\end{figure}

\clearpage
\subsection{Comparison}

This now compares all the data from simulations with different parameters. We
have run all simulations with a window of 2 units SSRCalc, and then variable
Q1/Q3. All notations of the form 50Da, 10ppm mean that Q1 was set to 50 Da and
Q3 to 10ppm.

\begin{table}[h]

\centering
\caption[]
{\textbf{Percentage of non-unique transitions}
Comparison of the methods explored above using the percentage of non-unique
transitions of the total number of transitions.
}
\label{tab:comp_nonunique}

\begin{tabular}{ l  c }
\maketablespace
Method &Non-unique transitions / \%  \\
\toprule
SRM (0.7 Da, 1 Da) & 24.7 \\
\midrule
1 Da, 10 ppm&  \phantom{1}5.1 \\
9 Da, 10 ppm&  18.3\\
15 Da, 10 ppm& 26.5\\
25 Da, 10 ppm& 35.7\\
50 Da, 10 ppm& 51.0\\

\end{tabular}
\end{table}

See Figure~\ref{fig:400range.comp} and Figure~\ref{fig:400range.comp_cum} for
the individual distributions (cumulative and single). See
Figure~\ref{fig:400range.comp_cum_all} for a comparison all the cumulative
distributions. The less space under the curve, the better.

\begin{figure}

\center
\subfloat[1][Traditional: 0.7 Da, 1 Da]{\includegraphics[width=0.3\textwidth, angle=-90]{/home/hroest/srm_clashes/results/pedro/yeast_True_False_20_7_10_range400to1200_comp.pdf}} 
\subfloat[1][Traditional: 1 Da, 1 Da]{\includegraphics[width=0.3\textwidth, angle=-90]{/home/hroest/srm_clashes/results/pedro/yeast_True_False_20_10_10_range400to1200_comp.pdf}} \\
%\subfloat[1][1Da, 10ppm]{\includegraphics[width=0.3\textwidth, angle=-90]{/home/hroest/srm_clashes/results/pedro/yeast_True_False_20_10_100ppm_range400to1200_comp.pdf}} \\
\subfloat[1][9Da, 10ppm]{\includegraphics[width=0.3\textwidth, angle=-90]{/home/hroest/srm_clashes/results/pedro/yeast_True_False_20_90_100ppm_range400to1200_comp.pdf}} 
\subfloat[1][15Da, 10ppm]{\includegraphics[width=0.3\textwidth, angle=-90]{/home/hroest/srm_clashes/results/pedro/yeast_True_False_20_150_100ppm_range400to1200_comp.pdf}} \\
\subfloat[1][25Da, 10ppm]{\includegraphics[width=0.3\textwidth, angle=-90]{/home/hroest/srm_clashes/results/pedro/yeast_True_False_20_250_100ppm_range400to1200_comp.pdf}} 
\subfloat[1][50Da, 10ppm]{\includegraphics[width=0.3\textwidth, angle=-90]{/home/hroest/srm_clashes/results/pedro/yeast_True_False_20_500_100ppm_range400to1200_comp.pdf}} \\

\caption{ \textbf{Percentage unique transitions in the tryptic yeast peptidome.}
Comparison of the histograms.  }
\label{fig:400range.comp}
\end{figure}
\begin{figure}

\center
\subfloat[1][Traditional: 0.7 Da, 1 Da]{\includegraphics[width=0.3\textwidth, angle=-90]{/home/hroest/srm_clashes/results/pedro/yeast_True_False_20_7_10_range400to1200_cum.pdf}} 
\subfloat[1][Traditional: 1 Da, 1 Da]{\includegraphics[width=0.3\textwidth, angle=-90]{/home/hroest/srm_clashes/results/pedro/yeast_True_False_20_10_10_range400to1200_cum.pdf}} \\
%\subfloat[1][1Da, 10ppm]{\includegraphics[width=0.3\textwidth, angle=-90]{/home/hroest/srm_clashes/results/pedro/yeast_True_False_20_10_100ppm_range400to1200_cum.pdf}} \\
\subfloat[1][9Da, 10ppm]{\includegraphics[width=0.3\textwidth, angle=-90]{/home/hroest/srm_clashes/results/pedro/yeast_True_False_20_90_100ppm_range400to1200_cum.pdf}} 
\subfloat[1][15Da, 10ppm]{\includegraphics[width=0.3\textwidth, angle=-90]{/home/hroest/srm_clashes/results/pedro/yeast_True_False_20_150_100ppm_range400to1200_cum.pdf}} \\
\subfloat[1][25Da, 10ppm]{\includegraphics[width=0.3\textwidth, angle=-90]{/home/hroest/srm_clashes/results/pedro/yeast_True_False_20_250_100ppm_range400to1200_cum.pdf}} 
\subfloat[1][50Da, 10ppm]{\includegraphics[width=0.3\textwidth, angle=-90]{/home/hroest/srm_clashes/results/pedro/yeast_True_False_20_500_100ppm_range400to1200_cum.pdf}} \\

\caption{ \textbf{Percentage unique transitions in the tryptic yeast peptidome.}
Comparison of the cumulative distributions.  }
\label{fig:400range.comp_cum}
\end{figure}
\begin{figure}

\center
\includegraphics[width=0.7\textwidth, angle=-90]{/home/hroest/srm_clashes/results/multiple_csv/cumm_plot.pdf}

\caption{ \textbf{Percentage unique transitions in the tryptic yeast peptidome.}
Comparison of the cumulative distributions using different Q1 window sizes and
10 ppm accuracy against two SRM standards: 
a) 0.7 Da Q1 and 1.0 Da Q3 window and
b) 1.0 Da Q1 and 1.0 Da Q3 window.
The less space under the curve, the less peptides have a low amount of unique
transitions. Thus }
\label{fig:400range.comp_cum_all}
\end{figure}

\clearpage
\subsubsection{Sharing more than 2 transitions}

By looking at Table~\ref{tab:yeast_doublecollisions} 
we see that the number of interactions
between 2 peptides that involves several collisions decreases exponentially
(about one order of magnitude for each new number of collisions). This is not
true on peptide level, if we are only interested in the maximum number of
collisions one peptide has with any other peptide (since we apply the max
function), see Table~\ref{tab:yeast_doublecollisions_peptide}.

\begin{table}[h]

\centering
\caption[Collisions between peptides]
{\textbf{Collisions between peptides} This show all possible interactions
between 2 peptides (ion A with peptide B). It is not symmetric (e.g. not the
same as ion B with peptide A) since peptide A in the background has more
transitions than ion A (which has only 2+ parent / 1+ fragment).\\
Top: Data from 50 Da, 100ppm; Below: Data from 20 Da, 20ppm
}
\label{tab:yeast_doublecollisions}

\begin{tabular}{ r c c }
\maketablespace
Collisions & Number of interactions & Percentage \\
\toprule

1 & 8\,099\,495 & 87.466 \\
2 & 1\,020\,376 & 11.019 \\
3 & \phantom{1\,}122\,025 & \phantom{1}1.317 \\
4 & \phantom{1\,1}14\,806 & \phantom{1}0.159 \\
5 & \phantom{1\,11}2\,094 & \phantom{1}0.022 \\

\midrule

1 & \phantom{1\,}991\,396 & 90.037 \\
2 & \phantom{1\,1}95\,779 & \phantom{1}8.698 \\
3 & \phantom{1\,1}10\,880 & \phantom{1}0.988 \\
4 & \phantom{1\,11}1\,879 & \phantom{1}0.170 \\
5 & \phantom{1\,111\,}470 & \phantom{1}0.042 \\


\end{tabular}
\end{table}

\begin{table}[h]

\centering
\caption[Collisions between peptides]
{\textbf{Collisions between peptides} This show all possible interactions
between 2 peptides (ion A with peptide B). \\
Here we look at the top number of collisions that a given peptide A has with
any other peptide. \\
Top: Data from 50 Da, 100ppm; Below: Data from 20 Da, 20ppm
}
\label{tab:yeast_doublecollisions_peptide}

\begin{tabular}{ r c c }
\maketablespace
Minum number of Collisions & Number of Peptides \\
\toprule

1 & \phantom{}132\,460 \\
2 & \phantom{}122\,301 \\
3 & \phantom{}70\,241 \\
4 & \phantom{}15\,726 \\
5 & 3\,245 \\
6 & 1\,234 \\
7 & \phantom{1\,}736 \\

\midrule

1 & \phantom{}120\,118 \\
2 & \phantom{}60\,545 \\
3 & \phantom{}12\,049 \\
4 & 2\,849 \\
5 & 1\,102 \\
6 & \phantom{1\,}642 \\
7 & \phantom{1\,}419 \\


\end{tabular}
\end{table}



\clearpage


\section{Sharing more than 2 transitions}
To answer the question how often a double collision occurs, we look at
\url{count_pair_collisions} where we see with 50 Da and 100 ppm and a total of
134410 parent ions. We have 122301 that have another peptide with at least 2
shared transitions, 70241 with at least 3 shared transitions.

\begin{verbatim}
[132460, 122301, 70241, 15726, 3245, 1234, 736, 0, 0, 0]
\end{verbatim}

This is very bad, lets see what 20Da/20ppm gives us. We have 120118 peptides
sharing 1 or more transitions with another peptide, 60545 peptides sharing 2 or
more transitions and 12049 peptides sharing 3 or more transitions.

\begin{verbatim}
[120118, 60545, 12049, 2849, 1102, 642, 419, 0, 0, 0]
\end{verbatim}


I wrote some code to see what will happen if we dont have RT information. It
looks really bad: 
Yeast
4.77 transitions per 0.7 Th x 1.0 Th in [400,1200]
1.7 \% unique transitions
Human
9.73 transitions per 0.7 Th x 1.0 Th in [400,1200]
0.2 \% unique transitions
up to 600 transitions in one bin 
\newline
Thus even better when we do have RT information, we gain about a factor of 45 in yeast

\section{Add 3 isotope traces}
0.7 / 0.7 and 2 SSRCalc units in SRM we get Nonunique / Total transitions : 769176 / 1867985 = 0.411767760448
So we have 40\% non unique transitions. Compared to 24\% when we use 0.7 and 1 (so an even bigger window!)

looking at 25Da and 10ppm we get Nonunique / Total transitions : 704479 / 1867985 = 0.377133114024
this is only an increase of 5~\% from 35~\% to 37~\% whereas the increase in SRM setting was much bigger.

\section{Retention time validation}

608 peptides were measured in SRM mode in 27 runs, each run contained 11
standard peptides that span the whole RT range. These were used to normalize
the RT values. So in total 184'528 interactions were considered.

The data was from  \url{A_D100819_SPLASMAVER_TRID1521_xxx} where \url{xxx} is
the variable part (27 different files). These were SRM-runs with RT peptides
spiked in.

The first question to ask is, how reproducible the RT peptides are across the
runs. Thus the coefficient of determination $R^2$ was computed for each pair of
those 27 runs. The coefficients are plotted in Figure~\ref{fig:Rsquared}.


\begin{figure}
\includegraphics[width=\textwidth]{/home/hroest/srm_clashes/results/rtvalid/Rsquared.pdf}
\caption{ \textbf{Rsquared values for the 11 RT peptides in 27 MS runs.}
All against all comparison of the retention times of 11 standard peptides in 27
MS runs. From each run, the retention times of the 11 RT standard peptides were
correlated against the retention times of the 11 peptides in a different run.
The resulting Rsquared value from the correlation was recorded and is plotted.
In total 351 correlation analyses were performed.  }
\label{fig:Rsquared}
\end{figure}

\begin{figure}
\includegraphics[width=\textwidth]{/home/hroest/srm_clashes/results/rtvalid/SSRCalc_ROC.pdf}
\caption{ \textbf{ROC curve for SSRCalc predictions.}
A certain window in retention time was chosen (e.g. 1 minute, top curve) and
then for all values of SSRCalc between 0 and 10 arbitrary units, all
interactions within that window were considered.
TP were peptides that fell into the SSRCalc window as well as the real RT
window. FN were peptides that were in the original RT window but not in the
SSRCalc window.  The 5 curves show different RT windows with the corresponding
AUC (area under the curve) while a random classification is shown in green.}
\label{fig:SSRCalcROC}
\end{figure}


\begin{figure}

\center
\subfloat[1][]{\includegraphics[width=0.4\textwidth]{/home/hroest/srm_clashes/results/rtvalid/boxplot_intersect.pdf}} 
\subfloat[1][]{\includegraphics[width=0.4\textwidth]{/home/hroest/srm_clashes/results/rtvalid/boxplot_m.pdf}}  \\
\subfloat[1][]{\includegraphics[width=0.4\textwidth]{/home/hroest/srm_clashes/results/rtvalid/corr_corrected.pdf}} 
%\subfloat[1][]{\includegraphics[width=0.4\textwidth]{/home/hroest/srm_clashes/results/rtvalid/difference_correlation.pdf}} \\


\caption{ \textbf{Correlation plots.}
a) and b) shows the spread of the regression variables m and c when fitting the
RT peptides of the 27 runs onto the corresponding SSRCalc values. c) and d)
show the correlation of SSRCalc to rention time or the
correlation of the calculated differences, respectively. Normalization occured
through the different runs using the RT peptides. }
\label{fig:400ra.comp}
\end{figure}


\begin{figure}
\includegraphics[width=\textwidth]{/home/hroest/srm_clashes/results/rtvalid/difference_histogram.pdf}
\caption{ \textbf{Difference between retention time and SSRCalc.}
Histogram showing the deviation of the real difference in retention time and
the predicted difference in retention time by SSRCalc. }
\label{fig:DiffHist}
\end{figure}


\section{speed}

on a laptop the following command A) takes

\verbatim{
python run_integrated.py 123456789 700 710 --peptide_table=hroest.srmPeptides_yeast --max_uis 5 -i 3 --q1_window=1 --q3_window=1 --ssrcalc_window=10
}

takes 8.8 s with an average speed of 760 - 770 peptides/s. this value
increased from 710- 720
peptides/s after using std\dots string in all the structures. 
When using a std\dots vector approach to store the precursors, the number goes
down to 61- 630. if we remove it, we go back to 750. This is a drop of almost
20\% in performance.


\verbatim {
time python run_integrated.py 123456789 700 710 --peptide_table=hroest.srmPeptides_yeast --max_uis 5 -i 3 --q1_window=3.5 --q3_window=2 --ssrcalc_window=20  
}


\end{document}
